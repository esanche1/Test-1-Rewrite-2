\documentclass[12pt]{article}
 
\usepackage[margin=1in]{geometry}
\usepackage{amsmath,amsthm,amssymb}
 
\newcommand{\N}{\mathbb{N}}
\newcommand{\R}{\mathbb{R}}
\newcommand{\Z}{\mathbb{Z}}
\newcommand{\Q}{\mathbb{Q}}
 
\newenvironment{theorem}[2][Theorem]{\begin{trivlist}
\item[\hskip \labelsep {\bfseries #1}\hskip \labelsep {\bfseries #2.}]}{\end{trivlist}}
\newenvironment{lemma}[2][Lemma]{\begin{trivlist}
\item[\hskip \labelsep {\bfseries #1}\hskip \labelsep {\bfseries #2.}]}{\end{trivlist}}
\newenvironment{exercise}[2][Exercise]{\begin{trivlist}
\item[\hskip \labelsep {\bfseries #1}\hskip \labelsep {\bfseries #2.}]}{\end{trivlist}}
\newenvironment{problem}[2][Problem]{\begin{trivlist}
\item[\hskip \labelsep {\bfseries #1}\hskip \labelsep {\bfseries #2.}]}{\end{trivlist}}
\newenvironment{question}[2][Question]{\begin{trivlist}
\item[\hskip \labelsep {\bfseries #1}\hskip \labelsep {\bfseries #2.}]}{\end{trivlist}}
\newenvironment{corollary}[2][Corollary]{\begin{trivlist}
\item[\hskip \labelsep {\bfseries #1}\hskip \labelsep {\bfseries #2.}]}{\end{trivlist}}
 
\begin{document}
 
\title{Assignment 11}
\author{Ed Sanchez\\ 
Math 200: Techniques of Mathematical Proofs}
 
\maketitle
 
\begin{problem}{9.9}
\text{ }\\
\end{problem}
 
\begin{enumerate}
\item R1 is not a function
\\
\item R2 is not a function
\\
\item R3 is a function
\\
\item R4 is a function
\\
\item R5 is not a function
\end{enumerate}
 
\begin{problem}{9.10}
\text{ }
\begin{enumerate}
\item $g(\mathbb{Z})=\{...-7,-3,1,5,9...\}$
\\
g(E)=\{4(2n)+1\}
\\
g(E)=\{...,-15,-7,1,9,17,...\}
\\
\item $g^{-1}(\mathbb{N})=\{ \frac{r-1}{4}: r \in \mathbb{N}\}$
\\
$g^{-1}(\mathbb{N})=\{0,\frac{1}{4},\frac{1}{2},\frac{3}{4}...\}$
\\
$g^{-1}(D)=\{\frac{n-1}{2}:n \in \mathbb{Z}\}$
\\
$g^{-1}(D)=\{...,\frac{-3}{2},-1,\frac{-1}{2}...\}$
\end{enumerate}
\end{problem}


\begin{problem}{9.11}
\text{ }
\end{problem}

\begin{enumerate}
\item F(C) = \{ $x^2 \geq 1$:$x \in \mathbb{R}$,$x\geq 1$\}
\\
$F^{-1}$(C)=$C \cup \{x \in \mathbb{R} : -x \in C\}$
\\
$F^{-1}$(D)=$( 0, \infty ) \cup (-\infty , 0)$
\\
$F^{-1}(\{1\})= \{ 1, -1\} $
\\
\item F(C) =$[ 0, \infty\}$
\\
$F^{-1}$(C)=$\{ e^x : x \in \mathbb{R}, x \geq 1\}$
\\
$F^{-1}$(D)=$\{e^x: x \mathbb{R}^+ \}$
\\
$F^{-1}(\{1\})= \{ e \} $
\\
\item F(C) =$[ e, \infty )$
\\
$F^{-1}$(C)=$[0,\infty)$
\\
$F^{-1}$(D)=$(- \infty, \infty)$
\\
$F^{-1}(\{1\})= \{ 0 \} $
\\
\item
F(C) =$[ -1, 1 ]$
\\
$F^{-1}$(C)=$\{ sin ^{-1} 1\}$
\\
$F^{-1}$(D)=$\{ sinc^{-1}: x \mathbb{R}^+ \}$
\\
$F^{-1}(\{1\})= \{ sin ^{-1} 1\}$
\\
\item
$F(C) =( - \infty, 1 ]$
\\
$F^{-1}$(C)=$\{1\}$
\\
$F^{-1}$(D)=$(0,2)$
\\
$F^{-1}(\{1\})=\{ 1 \}$
\end{enumerate}

\begin{problem}{9.23}
\text{ }\\
The claim is valid because F(a) = \{a\}
\end{problem}


\begin{problem}{9.24}
\text{ }
\end{problem}
\begin{enumerate}
\item Recognize $0,-4\in\mathbb{R}$. Since f(0)=9 and f(-4)=9, f is not one-to-one.
\item When we isolate x, we find that $x=\sqrt[y-5]-2$. Yet for every $y\in \mathbb{R}$, $x \not \in \mathbb{R}$. For example, if y=4,$\sqrt[y-5]-2 \not \in \mathbb{R}$. F is not onto.
\end{enumerate}


\begin{problem}{9.25}
\text{ }\\
There is a function from $\mathbb{R}$ to $\mathbb{R}$ that is onto but not one to one. For example, F(x) = x((x+1)(x-1)). Since f(0)=f(1), f is not one to one. Furthermore, Since the function is continuous and differentiable, f is onto.  
\end{problem}

\begin{problem}{9.31}
\text{ }\\
Claim: F is well defined
\end{problem}
\begin{proof}
Assume that f: $\mathbb{Z}_5 \Rightarrow \mathbb{Z}_5$ where f([a])=[2a+3]. Now assume that [a]=[b] for $[a],[b]\in\mathbb{Z}_5$. Therefore, for $m\in\mathbb{Z}$ $a-b=5m$. This means
\begin{align}
(2a+3)-(2b+3)&=2(a-b)
\\ \nonumber
&=5(2m)
\\ \nonumber
\end{align}
Since $2m \in \mathbb{Z}$, $ 5|(2a+3)-(2b+3)$. Hence, [2a+3]=[2b+3] so f(a) = f(b). The function is well defined. 
\end{proof}
For the function,f, to be bijective, each distinct image of f must have a unique pre-image and vise versa. Since $\mathbb{Z}_5$ has 5 classes, we need to show that all 5 classes have distinct images. 
\begin{align}
f([0])&=[2(0)+3]
\\ \nonumber
&=[3]
\\ 
f([1])&=[2(1)+3]
\\ \nonumber
&=[5]
\\ \nonumber
&=[0]
\\
f([2])&=[2(2)+3]
\\ \nonumber
&=[7]
\\ \nonumber
&=[2]
\\
f([3])&=[2(3)+3]
\\ \nonumber
&=[9]
\\ \nonumber
&=[4]
\\
f([4])&=[2(4)+3]
\\ \nonumber
&=[11]
\\ \nonumber
&=[1]
\end{align}
Since every distinct element has a distinct element, the function is injective. Since $f(\mathbb{Z}_5)=\mathbb{Z}_5$, the function is also surjective. Therefore, the function is bijective. 
\\
\begin{problem}{9.43}
\text{ }\\
Claim:If g $\circ$ h is one-to-one, then f is one-to-one.
\end{problem}
\begin{proof}
Assume $g \circ f: A \rightarrow C$ is one-to-one. Since$g \circ f$ is one-to-one, $a_1,a_2 \in A$. Therefore,$g \circ f (a_1)=g \circ f (a_2)\Rightarrow a_1 = a_2$.Therefore, f is one to one.
\end{proof}

\begin{proof}
Assume that the function, f, is not one to one so g $\circ$ f is not one to one. That means that f must at least one element that has an identical image in the codomian. Assume that $a,b \in A, a \not = b$ and f(a)=f(b). But, we find that $g \circ f (a)=g \circ f (b)$. Therefore, g $\circ$ f is not one to one and the result holds by the method of contrapositive. 
\end{proof}

\begin{proof}
Assume, to the contrary, that there exist a function such that g $\circ$ f is injective and f is not one to one. Since f is not one to one $a \not = b$ but f(a)=f(b). However, we find that $g \circ f (a)=g \circ f (b)$. This contradicts the assumption that  g $\circ$ f is not one to one and that f is one to one if g $\circ$ f is one to one by contradiction. 
\end{proof}

\begin{proof}
In order to prove if g $\circ$ f is one to one, then g is not one to one, we must show one counterexample. First, we must start by defining A,B and C. A = \{ 1,2,3\} , B= \{4,5,6,7\} and C=\{8,9,10\}.\\
Thus, g $\circ$ f =\{(1,8),(2,9),(3,10)\}. So g $\circ$ f is one-to one. g=\{(4,8),(5,9),(6,10),(7,10)\} Yet g contains elements from B that have some from the same image in C.Specifically, 6 and 7. Therefor g is not one to one. Therefore, if g $\circ$ f is one to one, then g is not one to one. 
\end{proof}
\end{document}
