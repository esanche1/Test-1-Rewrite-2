\documentclass[12pt]{article}
 
\usepackage[margin=1in]{geometry}
\usepackage{amsmath,amsthm,amssymb}
 
\newcommand{\N}{\mathbb{N}}
\newcommand{\R}{\mathbb{R}}
\newcommand{\Z}{\mathbb{Z}}
\newcommand{\Q}{\mathbb{Q}}
 
\newenvironment{theorem}[2][Theorem]{\begin{trivlist}
\item[\hskip \labelsep {\bfseries #1}\hskip \labelsep {\bfseries #2.}]}{\end{trivlist}}
\newenvironment{lemma}[2][Lemma]{\begin{trivlist}
\item[\hskip \labelsep {\bfseries #1}\hskip \labelsep {\bfseries #2.}]}{\end{trivlist}}
\newenvironment{exercise}[2][Exercise]{\begin{trivlist}
\item[\hskip \labelsep {\bfseries #1}\hskip \labelsep {\bfseries #2.}]}{\end{trivlist}}
\newenvironment{problem}[2][Problem]{\begin{trivlist}
\item[\hskip \labelsep {\bfseries #1}\hskip \labelsep {\bfseries #2.}]}{\end{trivlist}}
\newenvironment{question}[2][Question]{\begin{trivlist}
\item[\hskip \labelsep {\bfseries #1}\hskip \labelsep {\bfseries #2.}]}{\end{trivlist}}
\newenvironment{corollary}[2][Corollary]{\begin{trivlist}
\item[\hskip \labelsep {\bfseries #1}\hskip \labelsep {\bfseries #2.}]}{\end{trivlist}}
 
\begin{document}
 
\title{In Class Presentation 1}
\author{Ed Sanchez\\ 
Math 366: Techniques of Mathematical Proofs}
 
\maketitle
 
\begin{problem}{: Worksheet 3; Question 4}
\text{ }\\
Claim: If $<\gamma\'(t),u>$ $\leq$ $|\gamma'(t)|$, then $|q-p|\leq$$\int_{a}^{b} |\gamma'(t)| dt$ for $\gamma(a)=p$ and $\gamma(b)=q$.
\end{problem}
\begin{proof}
Assume that $<\gamma\'(t),u>$ $\leq$ $|\gamma'(t)|$. Using the Cauchy Schwartz inequality, one can claim the following. 
\begin{align}
\\ \nonumber
<\gamma\'(t),u> \leq |\gamma'(t)| 
\\ \nonumber
<\gamma\'(t),u> \leq |\gamma'(t)||u|
\\ \nonumber
|\gamma'(t)||u|=|\gamma\'(t)|
\end{align}
The last step from (1) can be completed because the norm of a unit vector is equivalent to 1. Next, we will expand the left side of the inequality. 
\begin{align}
\\ \nonumber
&\int \gamma'(t) \bullet u \ dt 
\\ \nonumber
&= \int \gamma'(t)_{1}u_{1} dt + \int \gamma'(t)_{2}u_{2} dt 
\\ \nonumber
&= u\int \gamma'(t)_{1} dt + \int \gamma'(t)_{2} dt
\\ \nonumber 
&= u\int \gamma'(t) dt
\end{align}
\begin{comment}
%\int f_{1}(t)+f_{2}(t) dt =\int f_{1}(t) dt + \int f_{2}(t) dt
%\int
\end{comment}
From here, one can apply the bounds of the integral and continue to solve the inequality
\begin{align}
\\ \nonumber
\int_{a}^{b} \gamma'(t) dt \bullet u \leq \int_{a}^{b} |\gamma'(t)| dt
\\ \nonumber
<\gamma(b)-\gamma(a), u> \leq \int_{a}^{b} |\gamma'(t)| dt
\\ \nonumber
<q-p, u> \leq \int_{a}^{b} |\gamma'(t)| dt
\end{align}
By taking $u=\frac{(q-p)}{|q-p|}$, one can see...
\begin{align}
\\ \nonumber
<q-p, \frac{(q-p)}{|q-p|}> \leq \int_{a}^{b} |\gamma'(t)| dt 
\\ \nonumber
(q-p) \bullet \frac{(q-p)}{|q-p|} \leq \int_{a}^{b} |\gamma'(t)| dt
\\ \nonumber
\frac{|q-p|^2}{|q-p|} \leq \int_{a}^{b} |\gamma'(t)| dt
\\ \nonumber
|q-p| \leq \int_{a}^{b} |\gamma'(t)| dt
\\ \nonumber
\end{align}
Interpreting the results, one can conclude that a straight line between  will be equal to or shorter than the arc length of any other curve.
\end{proof}




\end{document}
