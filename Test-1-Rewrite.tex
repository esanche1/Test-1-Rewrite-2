\documentclass[12pt]{article}
 
\usepackage[margin=1in]{geometry}
\usepackage{amsmath,amsthm,amssymb}
 
\newcommand{\N}{\mathbb{N}}
\newcommand{\R}{\mathbb{R}}
\newcommand{\Z}{\mathbb{Z}}
\newcommand{\Q}{\mathbb{Q}}
 
\newenvironment{theorem}[2][Theorem]{\begin{trivlist}
\item[\hskip \labelsep {\bfseries #1}\hskip \labelsep {\bfseries #2.}]}{\end{trivlist}}
\newenvironment{lemma}[2][Lemma]{\begin{trivlist}
\item[\hskip \labelsep {\bfseries #1}\hskip \labelsep {\bfseries #2.}]}{\end{trivlist}}
\newenvironment{exercise}[2][Exercise]{\begin{trivlist}
\item[\hskip \labelsep {\bfseries #1}\hskip \labelsep {\bfseries #2.}]}{\end{trivlist}}
\newenvironment{problem}[2][Problem]{\begin{trivlist}
\item[\hskip \labelsep {\bfseries #1}\hskip \labelsep {\bfseries #2.}]}{\end{trivlist}}
\newenvironment{question}[2][Question]{\begin{trivlist}
\item[\hskip \labelsep {\bfseries #1}\hskip \labelsep {\bfseries #2.}]}{\end{trivlist}}
\newenvironment{corollary}[2][Corollary]{\begin{trivlist}
\item[\hskip \labelsep {\bfseries #1}\hskip \labelsep {\bfseries #2.}]}{\end{trivlist}}
 
\begin{document}
 
\title{Test 1-Rewrite 2}
\author{Ed Sanchez\\ 
Math 200: Techniques of Mathematical Proofs}
 
\maketitle
 
\begin{problem}{1}
\text{ }\\
Claim: Prove that A $\subseteq$ B and B $\subseteq$ C and C $\subseteq$ A, and A = B and B = C. 
\end{problem}
 
\begin{proof}
Assume A,B and C are sets and A $\subseteq$ B, B $\subseteq$ C and C $\subseteq$ A. By definition, if A $\subseteq$ B, then all elements of the set B are elements of the set A. Also, since B $\subseteq$ C, all elements of the set C are elements in the set B. Thus, if all elements in the set C are also elements of the set A and all elements of the set A are also elements in the set , A $\subseteq$ B and B $\subseteq$ A. Based on the definition of set equality, two sets are equal if and only if they have the same elements. Therefore, since A has all the elements of B and B has all the elements of A, A = B. Similarly, since all elements of the set A are also elements of the set b and all elements of the set C are also elements of the set A, B $\subseteq$ C and C $\subseteq$ B. Thus, B and C have the same elements. Since A $\subseteq$ B, B $\subseteq$ C and C $\subseteq$ A, then A = B and B = C.   
\end{proof}
 
\begin{problem}{2}
\text{ }\\
 Claim: Let A be any set and $\emptyset$ be the empty set.   
\end{problem}
\begin{enumerate}
\item Prove that $\emptyset$ $\subset$ A.
\begin{proof}
Let A be any set and $\emptyset$ be the empty set. Assume, via contradiction, that $\emptyset$ $\not\subseteq$ A. Then there exists at least one element, X, in $\emptyset$ such that x $\not\in$ A. yet, there is no such x in $\emptyset$ since $\emptyset$ is the empty set. Hence, the assumption that $\emptyset$ $\not\subseteq$ A is false. Therefore, $\emptyset$ $\subseteq$ A for any set A. 
\end{proof}
\item Prove that the empty set is unique
\end{enumerate}
\begin{proof}
Assume A to be distinct empty set and $\emptyset$ is another distinct empty set. Assume, via contradiction, that the empty set is not unique. This would mean that A is an element of $\emptyset$. Yet, by definition of the empty set, the empty set has no elements. Hence, the assumption that A is an element of $\emptyset$ is false. Therefore, the empty set is unique.
\end{proof}

\begin{problem}{3}
\text{ }\\
Claim: Let P(x,y,z) be an open sentence, where the domains of x,y and z are A,B and C, respectively. Express the equation of $\forall$ x $\in$ A $\vee$ $\forall$ y $\in$ B, $\exists$ z $\in$ C, P(x,y,z) in symbols.
\end{problem}
 
$\exists$ x $\in$ A $\lor$ $\exists$ y $\in$ B, $\forall$ z $\in$ C, $\sim$ P(x,y,z)

\begin{problem}{4}
\text{ }\\
Let P(n): $n^2$-n+5 be an open sentence over a domain S. 
\end{problem}
\begin{enumerate}
\item Determine the Truth Values of the quantified statements: $\forall n \in S$, P(n) and $\exists z \in S$, $\sim P(n)$ and S = $\{1,2,3,4\}$
\begin{center}
\begin{tabular}{|c|c|c|}
\hline
$\forall n \in S$, P(n) & $\exists z \in S$ & $\forall n \in S$, P(n) and $\exists z \in S$\\
\hline True & False & False\\
\end{tabular}
\end{center}
\item Determine the Truth Values of the quantified statements: $\forall n \in S$, P(n) and $\exists z \in S$, $\sim P(n)$ and S = $\{1,2,3,4,5\}$ \newline
\begin{center}
\begin{tabular}{|c|c|c|}
\hline
$\forall n \in S$, P(n) & $\exists z \in S$ & $\forall n \in S$, P(n) and $\exists z \in S$\\
\hline False & True & False\\
\end{tabular}
\end{center}


\end{enumerate}
How are statements in (4.1) and (4.2) related? \newline \newline
4.1 and 4.2 are contradictory statements. Specifically, there is no way the statements could be true because if one component($\forall n \in S$, P(n)) is true, then the other ($\exists z \in S$, $\sim P(n)$) becomes false. Even when the domain is expanded by one element, the truth values switch.  

\begin{problem}{5}
\text{ }\\
Prove by contrapositive the following
\end{problem}
\begin{enumerate}
\item Let n $\in$ $\mathbb{Z}$. Prove that $(n+1)^2-1$ is even if and only if n is even.\newline \newline
Claim: Let n $\in$ $\mathbb{Z}$. Prove that $(n+1)^2-1$ is even if and only if n is even.
\begin{proof}
Assume that n $\in$ $\mathbb{Z}$ and n is odd. Then n = 2m+1 for some integer m. Consider the following, 
\begin{align}
(n+1)^2-1=&((2m+1)+1)^2-1\\
\nonumber
=&(2m+2)^2-1\\
\nonumber
=&2m^2+4m+3\\
\nonumber
=&2(m^2+2m+1)+1\\
\nonumber
\end{align}
Since $(m^2 +2m)$ is an integer, $2(m^2+2m)+3$ is of the form $2k+1$. Thus, $(n+1)^2-1$ is odd. \newline
Now assume that is $(n+1)^2-1$ is odd. This would mean that $(n+1)^2-1$ would have to be of the form $2k+1$. Based on the arithmetic from (1), the only way $(n+1)^2-1$ can be odd is if n is odd. Therefore, $(n+1)^2-1$ is even if and only if n is even.
\end{proof}
\item Let x $\in$ $\mathbb{Z}$. Prove that (3x+1) is even if and only if 5x-2 is odd.\newline
Claim: Let x $\in$ $\mathbb{Z}$. Prove that (3x+1) is even if and only if 5x-2 is odd.
\begin{proof}
Assume that 5x-2 is even and x $\in$ $\mathbb{Z}$. In order for 5x-2 to be even, x needs to even. This means x=2k for some integer k. Consider the following.
\begin{align}
5x-2=&5(2k)-2\\
\nonumber
=&10k-2\\
\nonumber
=&10k-2\\
\nonumber
=&2(5k)-1\\
\nonumber
\end{align}
Since 5k is an integer, 2(5k)-1 is odd. Furthermore, consider that 3x+1 will be odd if 5x-2 is odd. That is, if x is still even, then 3x+1 will be odd. Observe the following. 
\begin{align}
3x+1=&3(2k)+1\\
\nonumber
=&6k+1\\
\nonumber
=&2(3k)+1\\
\nonumber
\end{align}
Since 3k is an integer, 2(3k)+1 is odd too. Thus, 3x+1 is odd. Now consider if 3x+1 is even. Therefore, x will need to be odd in order for 3x+1 to be even. Consider the following.
\begin{align}
3x+1=&3(2k+1)+1\\
\nonumber
=&6k+4\\
\nonumber
=&2(3k+2)\\
\nonumber
\end{align}
Since 3k+2 is an integer, 3x+1 is even. Now understand that 5x-2 is odd from the following arithmetic. 
\begin{align}
5x-2=&5(2k+1)-2\\
\nonumber
=&10k+3\\
\nonumber
=&2(5k+1)+1\\
\nonumber
\end{align}
Again, since 5k+1 is an integer, 5x-2 is odd. Altogether, (3x+1) is even if and only if 5x-2 is odd. 
\end{proof}
\end{enumerate}

\begin{problem}{6}
\text{ }\\
Evaluate the proposed proof of the following result. If the proof is flawed indicate what is wrong and correct it.
\end{problem}
The problem with the proof is that it makes conflicting statements. Specifically, the proof says that\newline
\begin{center}
Since (x,y) $\in$ (A $\times$ C), it follows that x $\in$ A and y $\in$ C. Since (x,y)$\not\in$ (B X C), we have x $\not\in$ B. 
\end{center}
One cannot make the selective claim that X is not an element of B while also claiming that x is an element of A and y is an element of C. In order to fix this, one should say
\begin{center}
(x,y) $\in$ (A $\times$ C) but x $\not\in$ B. Therefore, x $\in$ (A-B) and y $\in$ C. Hence, x,y $\in$ (A-B) $\times$ C
\end{center}

\end{document}
